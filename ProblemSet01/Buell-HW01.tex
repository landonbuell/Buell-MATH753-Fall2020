% ================
% Landon Buell
% Mark lyon
% MATH 753.01 - HE01
% 4 Sept 2020
% ================

\documentclass[12pt,letterpaper]{article}
\usepackage[top=2.5cm,left=2.5cm,right=2.5cm]{geometry}
\usepackage{amsmath,amssymb}
\usepackage{physics}


\begin{document}

% ================================================================

\title{MATH 753.01 - Fall 2020 \\ Homework 01}
\author{Landon Buell}
\date{4 September 2020}
\maketitle

% ================================================================

\section*{Section 0.4}

% ================================================

\subsection*{Ex 1}
Identify which values of $x$ that there is subtraction of nearly equal numbers - Find an equivalent expression.

\begin{itemize}

\item[•] (a)
\begin{equation}
\frac{1 - \sec(x)}{\tan^2(x)} 
\end{equation}
The $\sec(x)$ function is very near $+1$ for values of $x \approx 2n\pi$, this would create subtraction of nearly equal numbers. We can use trigonometric rules:
\begin{equation}
\frac{1 - 1/\cos(x)}{\tan^2(x)} = 
\end{equation}

\item[•] (b)
\begin{equation}
\frac{1 - (1 - x)^3}{x}
\end{equation}
For $x \approx 0$, the numerator evaluated roughly to $1 - 1$, thus we have subtraction of nearly equal numbers. Additionally, this would likely raise a \textit{zero-division error} if $x = 0$ exactly. We can expand the numerator, and find the algebraic equivalent:
\begin{equation}
\frac{1 - (-x^3 + 3x^2 - 3x + 1)}{x} = \frac{x^3 - 3x^2 + 3x}{x} = x^2 - 3x + 3
\end{equation}

\item[•] (c)
\begin{equation}
\frac{1}{1 + x} - \frac{1}{1 - x}
\end{equation}
For $x \approx 0$, both terms will evaluate to approximately $1/1$. This would create subtraction of nearly equal numbers. 

\end{itemize}

% ================================================

\subsection*{Ex 2}
Find the roots of $x^2  + 3x - 8^{-14} = 0$ with three-digit accuracy.

% ================================================


\subsection*{Ex 4}
Evaluate $x\sqrt{x^2 + 17} - x^2$ where $x = 9^{10}$  to three decimals.

\paragraph*{}We multiply the function by it's conjugate:
\begin{equation}
x\sqrt{x^2 + 17} - x^2 = 
\frac{(x\sqrt{x^2 + 17} - x^2)}{1}\frac{(x\sqrt{x^2 + 17} + x^2)}{(x\sqrt{x^2 + 17} + x^2)} =
\frac{(x^2(x^2 + 17) - x^4)}{(x\sqrt{x^2 + 17} + x^2)}
\end{equation}

% ================================================================

\section*{0.5}

% ================================================

\subsection*{Ex 4}
Find the Taylor Polynomial of degree $2$ about the point $x = 0$ for the following functions:

\begin{itemize}

\item[•] (a)
\begin{equation}
f(x) = e^x^2
\end{equation}

\item[•] (b)
\begin{equation}
\f(x) = cos(5x)
\end{equation}

\item[•] (c)
\begin{equation}
f(x) = \ln(1 + x)
\end{equation}

\end{itemize}

% ================================================

\subsection*{Ex 6}

\begin{itemize}

\item[•] (a) Find the Taylor Polynomial for degree $4$ for $f(x) = x^-2$ about the point $x = 1$.

\item[•] (b) Use the result to approximate $f(0.9)$ and $f(1.1)$.


\end{itemize}

% ================================================

\subsection*{Ex 8}

\begin{itemize}

\item[•] (a) ) Find the degree 5 Taylor polynomial $P(x)$ centered at $x = 0$ for $f(x) = \cos(x)$.

\item[•] (b) Find an upper bound for the error in approximating $f(x) = \cos(x)$ for 
$x \in [-\frac{pi}{4}, +\frac{\pi}{4}]$ by $P(x)$.


\end{itemize}

% ================================================

\subsection*{Ex 9}
A Commonyl approximation of $f(x) = \sqrt{1+x}$ is $1 + \frac{1}{2}x$, when $x$ is small. Use  a degree $1$ Taylor Polynomial.

% ================================================================

\section*{1.1}

% ================================================

\subsection*{Ex 2}

% ================================================

\subsection*{Ex 4}

% ================================================================

\end{document}


